\documentclass[xcolor=dvipsnames]{beamer}
\usetheme[navigation]{UMONS}
\usepackage[utf8]{inputenc}
\usepackage{graphicx}
\usepackage{wrapfig}
\usepackage{caption}
\usepackage{hyperref}
\usepackage{url}
\usepackage{color}
\usepackage{tikz}
\usepackage{fourier, heuristica}
\usepackage{array, booktabs}
\usepackage{graphicx}
%\usepackage[x11names]{xcolor}
\usepackage{colortbl}
\usepackage{caption}
\DeclareCaptionFont{SkyBlue}{\color{SkyBlue}}

\newcommand{\foo}{\color{SkyBlue}\makebox[0pt]{\textbullet}\hskip-0.5pt\vrule width 1pt\hspace{\labelsep}}

\renewcommand{\arraystretch}{1.9}
\newcommand{\green}[1]{\textcolor{ForestGreen}{#1}}
\newcommand{\red}[1]{\textcolor{red}{#1}}

\title{Blockchain}
\author[G.Jérémy, D.Arman, L. Semih]{Gheysen Jérémy, Davidyan Arman, Locqueneux Semih}
\date{24 Avril 2018}
\institute[]{%
 Faculté des Sciences\\
  Université de Mons
  \\[2ex]
  \includegraphics[height=4ex]{UMONS}\hspace{2em}%
  \raisebox{-1ex}{\includegraphics[height=6ex]{UMONS_FS}}
}

\begin{document}

\maketitle

\section{Introduction}

\begin{frame}{Introduction}{Historique}
\vspace{-0.5cm}
	\begin{table}
		\renewcommand\arraystretch{1.4}\arrayrulecolor{SkyBlue}
		\vskip -1.5ex

		\begin{tabular}{@{\,}r <{\hskip 2pt} !{\foo} >{\raggedright\arraybackslash}p{10cm}}
		
			\toprule
			\addlinespace[1.5ex]
			1991 & Premier travail sur une \textbf{chaîne sécurisée de blocs} par \textit{Haber et 	Stornetta}. \\
			1992 & Incorporation des \textbf{Merkle Trees} au design par \textit{Bayer, Haber et Stornetta}. \\
			2008 & Première conceptualisation en cryptomonnaie -- \textbf{BitCoin} par 	\textit{Satoshi Nakamoto}.\\
			2014 & Taille de la blockchain Bitcoin : 20 GB -- BitCoin : 600\$. \\
			2017 & Taille de la blockchain Bitcoin : 100 GB -- BitCoin : 1500\$. \\
			2018 & Taille de la blockchain Bitcoin : 164 GB -- BitCoin : 8000\$. \\
			
		\end{tabular}
	\end{table}
	
\end{frame}

\begin{frame}{Introduction}{Définition}
\vspace{-1cm}
	\begin{center}
		\includegraphics[scale=0.1]{storage.png} 
	\end{center}
	
	\begin{center}
	La blockchain est une technologie de \textbf{stockage} et de \textbf{transmission d’informations}, \textbf{transparente}, \textbf{sécurisée}, et fonctionnant \textbf{sans organe central de contrôle}.\footnote{\hspace{5pt}Définition de Blockchain France}
	\end{center}
	
	\begin{columns}
    	\begin{column}{0.48\textwidth}
    		\begin{center}
    			\includegraphics[scale=0.25]{relay.png} 
    		\end{center}
    	\end{column}
    	
    	\begin{column}{0.48\textwidth}
    		\begin{center}
				\includegraphics[scale=0.25]{control.png} 
			\end{center}
    	\end{column}
	\end{columns}

% Liens  https://openclipart.org/detail/167738/filing-cabinet-overload
%		 http://worldartsme.com/relay-race-free-clipart.html#gal_post_57417_relay-race-free-clipart-1.jpg
%		 http://www.clipartpanda.com/clipart_images/security-bag-check-clip-art-58416511
\end{frame}

\begin{frame}{Introduction}{Objectifs}
	\vspace{-0.5cm}
	\begin{columns}
    	\begin{column}{0.48\textwidth}
    		\only<1-3>{
    		\begin{center}
    			\textbf{Durable\\}
    			\begin{figure}
    				\includegraphics[scale=0.1]{perenity.jpg} 
    			\end{figure}
    		\end{center}}
    	\end{column}
    	
    	\begin{column}{0.48\textwidth}
			\only<2-3>{    		
    		\begin{center}
    			\textbf{Infalsifiable\\}
    			\begin{figure}
					\includegraphics[scale=0.15]{commands.png} 
				\end{figure}
			\end{center}}
    	\end{column}
	\end{columns}
    		\only<3-3>{
    		\begin{center}
    			\textbf{Distribué\\}
    			\begin{figure}
    				\includegraphics[scale=0.35]{central.png} 
    			\end{figure}
    		\end{center}}
% https://everestalexander.files.wordpress.com/2015/11/moses10commandmentstrans.gif
% http://zmeeed.info/aztec-architecture/popular-aztec-architecture-ancient-aztec-architecture-later-by-the-aztecs-th/
% http://blog.yintercept.com/2011/11/i-found-following-image-on-wikipedia.html
\end{frame}

\section{Fonctionnement}

\begin{frame}{Fonctionnement}{Hachage cryptographique}
	Exemple avec l'algorithme \textbf{SHA-256} :
	\vspace{1cm}
	\begin{center}
		\includegraphics[scale=0.4]{hash.png} 
	\end{center}
\end{frame}

\begin{frame}{Fonctionnement}{Chaîne de blocs}
	\begin{center}
		\begin{figure}
			\includegraphics[scale=0.28]{blockk.png} 
		\end{figure}
	\end{center}
\end{frame}


\begin{frame}{Fonctionnement}{Merkle Tree}
	\only<1>{
	\begin{center}
		\includegraphics[scale=0.35]{merkle_tree_correct.png} 
	\end{center}}
	
	\only<2>{
	\begin{center}
		\includegraphics[scale=0.35]{merkle_tree_incorrect.png} 
	\end{center}}
\end{frame}


\begin{frame}{Fonctionnement}{Preuve de travail}
	TODO SEMIH
\end{frame}

\section{Smart Contract}

\begin{frame}{Smart Contract}
	
	
	\begin{center}
		Nick Szabo, \textit{Smart Contracts: Building Blocks for Digital Markets}, 1996
	\end{center}		
	
	\begin{figure}
		\centering
		\includegraphics[scale=.25]{smart_contract}
		\caption{[Public domain]}
	\end{figure}
	
	
\end{frame}

\begin{frame}{Exemple d'Assurance}

	\begin{figure}
		\centering
		\includegraphics[scale=.5]{insurance_contract}
		\caption{By draglet GmbH [CC BY-SA 4.0 (https://creativecommons.org/licenses/by-sa/4.0)], from Wikimedia Commons}
	\end{figure}

\end{frame}

\begin{frame}{Exemple de code de contrat}

	\begin{figure}
		\centering
		\includegraphics[scale=.35]{contract_example}
		\caption{An example smart contract on Ethereum. Source: https://www.ethereum.org/token}
	\end{figure}

\end{frame}

\begin{frame}{Ethereum Tokens}

	\begin{center}
		\color{blue}
		Smart Contract de Ethereum permet de créer des blockchains sur la base de leur blockchain
	\end{center}
	
	Exemple:
	\begin{figure}
	\includegraphics[scale=.35]{golem}
	\caption{source: page Github de Golem}
	\end{figure}

\end{frame}

\section{Conclusion}

\begin{frame}{Quelques Implementations Interessant de Blockchain}

	\begin{itemize}
		\item Estonie a completement digitalisé les données de leurs citvoyens.
		\item Regulation de Commerce des poisson grâce à Hyperledger Sawtooth.
		\item Testing de smart cities à Dubai.
		\item Des votes éléctornique à Moscou en 2014.
	\end{itemize}
	
\end{frame}

\end{document}
